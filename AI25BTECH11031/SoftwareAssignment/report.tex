\documentclass{article}
\usepackage[a4paper,margin=1.5cm]{geometry}
\usepackage{amsmath,amsfonts}
\usepackage{graphicx}
\usepackage{enumitem}
\usepackage{multicol}
\usepackage{titlesec}
\usepackage{lmodern}
\usepackage{fancyhdr}
\usepackage{hyperref}
\usepackage{multirow}
\usepackage[utf8]{inputenc}
\usepackage{float}
\usepackage{xparse}
\usepackage{diagbox}
\usepackage{adjustbox}
\usepackage{array}
\title{Software Project\\ Image Compression Using Truncated SVD}
\author{Shivam Sawarkar \\ \textbf{AI25BTECH11031}}

\begin{document}

\maketitle
\thispagestyle{empty}

\section*{1. Background}
A grayscale image can be represented as a matrix $A \in \mathbb{R}^{m \times n}$, 
where each entry $a_{ij}$ corresponds to the pixel intensity 
($0 =$ black, $255 =$ white$)$.

The \textbf{Singular Value Decomposition (SVD)} expresses $A$ as:
\[
A = U \Sigma V^T
\]
where:
\begin{itemize}
    \item $U$ and $V$ are orthogonal matrices,
    \item $\Sigma$ is a diagonal matrix with singular values in decreasing order.
\end{itemize}

A low-rank approximation of $A$ can be obtained by keeping only the top $k$ singular values:
\[
A_k = U_k \Sigma_k V_k^T
\]
This truncated version $A_k$ preserves most of the important image content while using far fewer values, forming the basis of image compression.

\section*{2. Summary of Gilbert Strang’s Video on SVD}
The lecture explains the Singular Value Decomposition (SVD) of a matrix, highlighting its importance in transforming any matrix into a product of orthogonal matrices and a diagonal matrix. \\ \\ 

In his lecture prof. Gilbert Strang says:
\begin{enumerate}
    \item SVD is the best factorization of a matrix, consisting of two orthogonal matrices and a diagonal matrix.
    \item The goal of SVD is to find orthonormal bases for the row and column spaces of a matrix, allowing for diagonalization.
    \item The eigenvectors of $A^TA$ provide the orthogonal matrix $V$ in SVD, while the eigenvectors of $AA^T$ provide the orthogonal matrix $U$.
    \item The diagonal entries of the matrix $\Sigma$ in SVD are the square roots of the eigenvalues from $A^TA$ or $AA^T$.
\end{enumerate}

\section*{3. Algorithm Explanation}
\subsection*{Mathematical Basis}

\subsection*{Steps}
\begin{enumerate}
    \item Read grayscale image in PGM format and store as matrix $A$ (convert in PGM if  it is JPG).
    \item Compute $A^TA$ and $AA^T$.
    \item Use Power Iteration to find top eigenvalues and eigenvectors.
    \item Construct $U, \Sigma, V^T$.
    \item Truncate to top $k$ singular values.
    \item Reconstruct $A_k = U_k \Sigma_k V_k^T$.
    \item Write $A_k$ as new PGM file.
    \item Give output as .pgm or .jpg
\end{enumerate}

\section*{4. Pseudocode}
\noindent
Input: Grayscale image file (.pgm) \\
Output: Reconstructed compressed image (.pgm)

\begin{enumerate}
\item Read PGM image $\rightarrow$ matrix $A[m][n]$ (Convert to PGM first if the input is JPG)
\item Compute $A^TA$
\item For $i = 1, 2, \dots, k$: \\
\hspace{1em} Use power iteration on $A^TA$ to find top-$k$ eigenvectors $V$ \\
\hspace{1em} Compute $\sigma_i = \sqrt{\lambda_i}$
\item Compute $U_i = \frac{A V_i}{\sigma_i}$
\item Form $A_k = U_k \Sigma_k V_k^T$
\item Write $A_k$ to output PGM file (or JPG file)
\item Compute error $\|A - A_k\|_F$
\end{enumerate}

\section*{5. Algorithm Comparison for Computing SVD}

The below table compares various algorithms for calculating based on there speed, accuracy, and ease of coding

\begin{table}[H]
\centering
\renewcommand{\arraystretch}{1.3}
\begin{tabular}{|p{3cm}|p{5.2cm}|p{2cm}|p{2cm}|p{2.2cm}|}
\hline
\textbf{Algorithm} & \textbf{Description} & \textbf{Speed} & \textbf{Accuracy} & \textbf{Ease of Coding} \\ 
\hline
Jacobi Method & Iterative Givens rotations to diagonalize the matrix; accurate but computationally heavy. & Slow & Very High & Moderate \\ 
\hline
Golub--Reinsch & Reduces the matrix to bidiagonal form using Householder reflections before SVD computation. & Fast & High & Complex \\ 
\hline
Power Iteration & Repeatedly estimates dominant eigenvectors to find top singular values (used for truncated SVD). & Medium & Moderate & Easy \\ 
\hline
Lanczos Method & Improves power iteration with orthogonalization, making it efficient for large or sparse matrices. & Very Fast & High & Complex \\ 
\hline
Randomized SVD & Uses random projections to approximate dominant singular subspace with less computation. & Medium & Moderate & Easy \\ 
\hline
\end{tabular}
\caption{Algorithms for calculating SVD}
\end{table}

\noindent
\textbf{Result:}  
The Jacobi method provides the highest accuracy but is computationally difficult. 
The Golub--Reinsch algorithm, which reduces the matrix to a bidiagonal form before performing SVD.  

For this project, I have chosen the Power Iteration method because:
\begin{itemize}
    \item It computes only the top $k$ singular values and vectors, which is sufficient for image compression.
    \item It is easy to implement from scratch in C without using external libraries.
    \item It is moderately accurate and easy to code.
\end{itemize}

\section*{6. Discussion}
\begin{itemize}
    \item Increasing $k$ improves quality but takes more time.
    \item Computation time grows with image size and number of singular values computed.
    \item The first few singular values capture most of the image data. Beyond a certain $k$, additional singular values contribute very less to visual improvement but greatly increase computation.
    \item The Power Iteration method performs well for truncated SVD, though convergence slows for close or repeated singular values.
\end{itemize}

\section*{7. Error Analysis}
The approximation accuracy is measured using the Frobenius norm:
\[
\|A - A_k\|_F = \sqrt{\sum_{i,j} (A_{ij} - A_{k,ij})^2}
\]
As $k$ increases, the error decreases because smaller singular values contribute less to the total image energy.

\section*{8. Conclusion}
The project demonstrates image compression using truncated SVD implemented entirely in C.  
Results show that:
\begin{itemize}
    \item Low-rank approximation reduces storage effectively.
    \item SVD captures the essential image features.
    \item There exists a clear trade-off between compression ratio and quality.
\end{itemize}

\end{document}
